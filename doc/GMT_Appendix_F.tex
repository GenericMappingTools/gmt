%------------------------------------------
%	$Id: GMT_Appendix_F.tex,v 1.6 2004-01-02 22:45:12 pwessel Exp $
%
%	The GMT Documentation Project
%	Copyright 2000-2004.
%	Paul Wessel and Walter H. F. Smith
%------------------------------------------
%
\chapter{Chart of octal codes for characters}
\index{Characters!octal}
\index{Octal characters}
\thispagestyle{headings}

\GMTfig[h]{GMT_App_F_stand+}{Octal codes and corresponding symbols for StandardEncoding fonts}

The characters and their octal codes in the Standard encoded fonts
are shown in Figure~\ref{fig:GMT_App_F_stand+}, while the characters and their octal codes in the ISOLatin1
encoded fonts are shown in Figure~\ref{fig:GMT_App_F_iso+}.   Dark gray areas signify codes reserved for
control characters.  In order to use all the extended characters (shown in the light gray boxes) you need to
set {\bf CHAR\_ENCODING} to Standard+ or ISOLatin1+ in your \filename{.gmtdefaults4} file\footnote{If you chose
SI units during the installation then the default encoding is ISOLatin1+, otherwise it is Standard+.}.
The chart for the Symbol (\GMT\ font number 12) character
sets are presented in Figure~\ref{fig:GMT_App_F_symbol} below. The octal code is obtained by appending the
column value to the $\backslash$?? value, e.g., $\partial$ is
$\backslash$266 in the Symbol font.  The euro currency symbol is $\backslash$240 in the Symbol font and will
print if your printer supports it (older printer's firmware will not know about the euro).

\GMTfig[h]{GMT_App_F_iso+}{Octal codes and corresponding symbols for ISOLatin1Encoding fonts}

\index{Symbol font}
\index{Font!symbol}

\GMTfig[h]{GMT_App_F_symbol}{Octal codes and corresponding symbols for the Symbol font}

The Pifont ZapfDingbats is available as \GMT\ font number 34 and
can be used for special symbols not listed above.  The various
symbols are illustrated in Figure~\ref{fig:GMT_App_F_dingbats}.

\GMTfig[h]{GMT_App_F_dingbats}{Octal codes and corresponding symbols for ZapfDingbats font}
