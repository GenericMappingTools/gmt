%------------------------------------------
%	$Id: GMT_Appendix_P.tex,v 1.1 2004-08-26 02:54:15 pwessel Exp $
%
%	The GMT Documentation Project
%	Copyright 2000-2004.
%	Paul Wessel and Walter H. F. Smith
%------------------------------------------
%

\chapter{Using both \gmt\ 3 and 4}
\index{\GMT, using 3 and 4}
\thispagestyle{headings}

Because \GMT\ 4 is backward compatible with the 3.4.x series yet maintains its parameters
and history in separate files (e.g., \filename{.gmtdefaults4} versus \filename{.gmtdefaults})
it is possible to install and use both versions on the same workstation.  To simplify such
setups we supply the utility \progname{gmtswitch} which simplifies switching back and forth
between any number of installed versions.  Place the \progname{gmtswitch} script in your
general executable path (not in one of the \GMT\ bin directories) and run it after you have
finished installing all \GMT\ versions of interest.  The first time you run \progname{gmtswitch}
it will try to find all the available versions installed on your file system.  The versions
found will be listed in the file \filename{.gmtversions} in your home directory; each line
is the full path to the \GMT\ root directory (e.g., /usr/local/GMT3.4.2).  You may
manually add or remove entries there at any time.  You are then instructed to make a few
changes to your environment (shell dependent):
\begin{enumerate}
\item Define the environmental variable {\bf GMTHOME} to point to {\bf \$HOME}/this\_gmt.
Here, \filename{this\_gmt} is a link that will be created by \progname{gmtswitch} to point
to one of the installed versions. 
\item Make sure {\bf \$GMTHOME}/bin is in your {\bf PATH}.
\end{enumerate}
Make those edits and logout and back in again.  The next time you run \progname{gmtswitch}
you will be able to switch between versions.  Typing \progname{gmtswitch} will list the
available versions and prompt you to choose one, whereas \progname{gmtswitch} {\it version}
will immediately switch to that version ({\it version} must be a piece of unique text making
up the full path to a verson, e.g., 3.4.2).  If you use tcsh or csh you may have to type
``rehash'' to initiate the path changes.
